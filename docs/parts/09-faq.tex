\par\noindent\rule{\textwidth\color{pniblue}}{0.4pt}
\section{Troubleshooting}
\label{section:faq}

\begin{enumerate}

    \item \label{faq:err404} \textbf{Problem:}
    After starting ALIIAS, a browser window with an error message pops up (e.g. Error 404).
    \newline
    \textbf{Solution:}
    Sometimes the browser page is opened faster than the local server of ALIIAS. simply refresh the browser page (e.g. by pressing 'F5'). 
    
    \item \label{faq:nobrowser} \textbf{Problem:}
    After starting ALIIAS, no browser window is opened.
    \newline
    \textbf{Solution:}
    Try manually opening the main page of ALIIAS: start the web browser of your choice (e.g. Chrome) and type
    \href{http://127.0.0.1:5000}{\color{pniblue}\underline{http://127.0.0.1:5000}} to the address bar.
    
    \item \label{faq:nostart} \textbf{Problem:}
    ALIIAS does not run at all.
    \newline
    \textbf{Solution:}
    Check if the hardware key is plugged in.
    
    \item \label{faq:ls_login} \textbf{Problem:}
    Unable to log in into LimeSurvey.
    \newline
    \textbf{Solution:}
    Check internet connection. In case of forgotten user name or password, contact \href{mailto:sfb289.survey@uni-due.de}{sfb289.survey@uni-due.de}.
    
    \item \label{faq:survey_link} \textbf{Problem:}
    Unable to open survey link generated by ALIIAS.
    \newline
    \textbf{Solution:}
    Check internet connection. Wait 10-20 seconds and refresh the survey page (F5). Sometimes the LimeSurvey server requires more time (10-20 sec) to set up the survey for the actual participant.
    
    
     \item \label{faq:exit} \textbf{Problem:}
    A black command line window (terminal) remains open after closing ALIIAS.
    \newline
    \textbf{Solution:}
    ALIIAS must be properly exited before closing the internet browser tab (refer to Step 7 in section \ref{section:sop_aliias} for details). If this did not happen and the terminal window remained open, simply closing it will close ALIIAS properly. WARNING: do not close the terminal window if you are still working in the current session of ALIIAS.
    
     \item \label{faq:lostkey} \textbf{Problem:}
    The hardware key for ALIIAS got lost/stolen.
    \newline
    \textbf{Solution:}
    It is possible anytime to add new keys and ban existing keys from ALIIAS. This requires administrator privileges, owned by the central scientific project Z03. Please contact Tamas Spisak (\href{mailto:tamas.spisak@uk-essen.de}{\color{pniblue}tamas.spisak@uk-essen.de}). Stolen keys do not mean any security risk, after being banned.
    
\end{enumerate}

\par\noindent\rule{\textwidth\color{pniblue}}{0.4pt}
