%\par\noindent\rule{\textwidth\color{pniblue}}{0.4pt}
\section{Example use-cases}
\label{use-case:one-session}
\par\noindent\rule{\textwidth\color{pniblue}}{0.4pt}
 \addcontentsline{toc}{subsubsection}{Single-session}
\textbf{1. Initial registration only (single-session use-case):}
If questionnaires are the only type of data collected or all assessments (e.g. questionnaires + behavioral experiment) are done in one session (i.e with starting ALIIAS once, for pseudonymization of one participant), the initial registration with ALIIAS might already be sufficient. The researcher can instantly obtain the invitation link to the survey and the obtained short ID can be used for subsequent assessments, directly following the initial registration (SOP in section \ref{section:sop_aliias}).

\par\noindent\rule{\textwidth\color{pniblue}}{0.4pt}
\addcontentsline{toc}{subsubsection}{Multi-session}
\textbf{2. "Batched" initial registration of many subjects (multi-session use-case):}
With the proposed pseudonymization procedure, the researcher has the opportunity to "batch" initial registrations, i.e. perform the initial registration of many participants in one longer session, based on the personal data previously collected (e.g. during phone interview). The assignment of the participants to specific surveys and the actual experiments (MRI, behavior) can take place at a later time point. In this case, ALIIAS can be used multiple times to repeatedly obtain the pseudonym for a given participant (SOP in section \ref{section:sop_aliias}). ALIIAS displays if the participant was previously registered to any of the surveys. If the experimenter has previously registered the participant, this feature can be used for error checking (see QC2 section \ref{section:qc} for details). (Typographical errors in the personal data result in a different pseudonym which have not been registered yet, see QC1 in section \ref{section:qc}).

\par\noindent\rule{\textwidth\color{pniblue}}{0.4pt}
 \addcontentsline{toc}{subsubsection}{Repeated measures}
\textbf{3. Repeated assessments (multi-session + repeated measures use-case):}
Use-case 2 is easy to adapt to repeated assessments (e.g. before and after treatment). Single projects can add arbitrary ”experimental tags” to the shortID (e.g. append ’-w2’ for 'week 2') or use the built in features of the given experimental equipment (e.g. multiple surveys in LimeSurvey, see section \ref{section:ls_setup} or separate ’program cards’ on the MRI console) to distinguish between measurements. 

\par\noindent\rule{\textwidth\color{pniblue}}{0.4pt}
 \addcontentsline{toc}{subsubsection}{Parallel-sessions}
\textbf{4. Simultaneous assessment of many subjects (parallel-sessions use-case):}
PseudoID can be run on multiple computers simultaneously without any further notice. The reproducibility and uniqueness of pseudonyms and the consistency of the LimeSurvey database are still guaranteed.

\par\noindent\rule{\textwidth\color{pniblue}}{0.4pt}
\addcontentsline{toc}{subsubsection}{Computer failure during session}
\textbf{5. Computer failure during experiment:}
If the experiment has to be restarted, PseudoID can be repeatedly used to obtain the short ID. In the case of failure of the computer on which PseudoID runs, a backup computer can be used to obtain the short ID (even if the initial registration and other pseudonymization sessions were not performed on that computer).

\par\noindent\rule{\textwidth\color{pniblue}}{0.4pt}
\addcontentsline{toc}{subsubsection}{Incidental finding}
\textbf{6. Incidental finding:} In case of incidental finding, the LimeSurvey database will be used to link the long ID to the short ID and re-identification is performed by PseudoID (on any computer), given the long ID and the pseudonymization secret stored on the hardware key (SOP in section \ref{section:sop_reidentification}). 

\par\noindent\rule{\textwidth\color{pniblue}}{0.4pt}
\section{Quality Checks}
\label{section:qc}
Typographical errors in the personal data will result in a different pseudonym. This might not be a serious problem if the pseudonym is obtained only once as, in this case, re-identification of this pseudonym will simply result in the mistyped (but still manually likable) personal data. (Use case \ref{use-case:one-session}).
However, if ALIIAS is used for obtaining the pseudonym for the same participant multiple times (Use cases 2,3 and 5), typographical errors result in different pseudonyms for the same person, which requires manual adjustment during data consolidation.
To avoid this, two quality check steps must to be performed.
\begin{itemize}
    \item \textbf{QC1:} Typographical errors are most critical at the initial registration. Therefore, before displaying the pseudonym, ALIIAS shows a preview page, where personal data must be carefully checked for mistakes. (Note, that ALIIAS internally converts some characters that can be source of ambiguity, see Table \ref{tab:typo}). The researcher must explicitly state that "All details are correct", before being able to obtain the pseudonym (see also label 'QC1' on Fig. \ref{fig:flowchart}).
    \item \textbf{QC2:} If the initial registration was correct and the researcher already assigned the participant to at least one LimeSurvey survey with ALIIAS (\textbf{recommended to do right at initial registration}), than at repeated pseudonymization of the same participant, the preview page of ALIIAS will display the surveys to which the participant was already assigned. This feature can be considered as an additional opportunity for quality checking: in case of typos the researcher will see a survey report being contradictory to the expectations. Namely, ALIIAS will report that no participant with the (mistyped) details was added to any of the surveys.
\end{itemize}

\begin{table}[h]
 \caption{Handling of potentially ambiguous characters in ALIIAS}
  \centering
  \begin{tabular}{l l}
    \toprule
    Input Character      & Automatically converted to      \\
    \midrule
    \midrule
    ä            & ae                 \\
    \midrule
    ü           & ue                \\
    \midrule
    ö           & oe                \\
    \midrule
    ß           & ss                \\
    \midrule
    whitespace          &              \\
    \_          &              \\
    ,          & omitted             \\
    (          &              \\
    )          &              \\
    .          &              \\
    \bottomrule
  \end{tabular}
  \label{tab:typo}
\end{table}
