%\par\noindent\rule{\textwidth\color{pniblue}}{0.4pt}
\section{Example use-cases}
\par\noindent\rule{\textwidth\color{pniblue}}{0.4pt}
\textbf{1. Initial registration only (one-session use-case):}
If only questionnaires are collected or all assessments are done in one session, the initial registration with PseudoID might already be sufficient. In this case, the new participant must be assigned to all necessary surveys in LimeSurvey and must be invited to all, during the initial registration. The short ID generated during the initial registration can be used for any additional subsequent assessments, directly following the initial registration.

\par\noindent\rule{\textwidth\color{pniblue}}{0.4pt}
\textbf{2. "Batched" initial registration of many subjects (multi-session use-case):}
With the proposed pseudonymization procedure, the researcher has the opportunity to "batch" initial registrations, i.e. perform the initial registration of many participants in one longer session, based on the personal data previously collected (e.g. during phone interview). The assignment of the participants to specific surveys and the actual experiments (MRI, behavior) can take place at a later time point. In this case, PseudoID can be used multiple times to repeatedly obtain the pseudonym for a given participant. In this case, a database query is performed that allows for checking for typos in the personal data and ensure that the proper short ID is obtained.

\par\noindent\rule{\textwidth\color{pniblue}}{0.4pt}
\textbf{3. Repeated assessments (multi-session + repeated measures use-case):}
Use-case 2 is easy to adapt to repeated assessments (e.g. before and after treatment).
For convenience, PseudoID provides a pre-defined subset of "experimental tags" (e.g. baseline, week1, week2, etc) which is appended to the short ID.

\par\noindent\rule{\textwidth\color{pniblue}}{0.4pt}
\textbf{4. Simultaneous assessment of many subjects (parallel-sessions use-case):}
PseudoID can be run on multiple computers simultaneously without any further notice. The reproducibility and uniqueness of pseudonyms and the consistency of the LimeSurvey database are still guaranteed.

\par\noindent\rule{\textwidth\color{pniblue}}{0.4pt}
\textbf{5. Computer failure during experiment:}
If the experiment has to be restarted, PseudoID can be repeatedly used to obtain the short ID. In the case of failure of the computer on which PseudoID runs, a backup computer can be used to obtain the short ID (even if the initial registration and other pseudonymization sessions were not performed on that computer).

\par\noindent\rule{\textwidth\color{pniblue}}{0.4pt}
\textbf{6. Incidental finding:} In case of incidental finding, the LimeSurvey database will be used to link the long ID to the short ID and re-identification is performed by PseudoID (on any computer), given the long ID and the pseudonymization secret stored on the hardware key. 